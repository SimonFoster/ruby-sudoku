\input cwebxmac
\datethis

\N0 1. Generalized exact cover.
This program implements the algorithm discussed in my paper about ``dancing
links.'' I hacked it together from the {\mc XCOVER} program that I wrote in
1994; I apologize for not having time to apply spit and polish.

Given a matrix whose elements are 0 or 1, the problem is to
find all subsets of its rows whose sum is at most~1 in all columns and
{\it exactly\/}~1 in all ``primary'' columns. The matrix is specified
in the standard input file as follows: Each column has a symbolic name,
either one or two or three characters long. The first line of input contains
the names of all primary columns, followed by `\.{\char"7C}', followed by
the names of all other columns.
(If all columns are primary, the~`\.{\char"7C}' may be omitted.)
The remaining lines represent the rows, by listing the columns where 1 appears.

The program prints the number of solutions and the total number of link
updates. It also prints every $n$th solution, if the integer command
line argument $n$ is given. A second command-line argument causes the
full search tree to be printed, and a third argument makes the output
even more verbose.

\Y\B\D$\\{max\_level}$\5
$\T{150}$\C{ at most this many rows in a solution }\par
\B\D$\\{max\_degree}$\5
$\T{1000}$\C{ at most this many branches per search tree node }\par
\B\D$\\{max\_cols}$\5
$\T{10000}$\C{ at most this many columns }\par
\B\D$\\{max\_nodes}$\5
$\T{1000000}$\C{ at most this many nonzero elements in the matrix }\par
\Y\B\8\&\#\&{include}\5
$\.{<stdio.h>}$\6
\8\&\#\&{include}\5
$\.{<ctype.h>}$\6
\8\&\#\&{include}\5
$\.{<string.h>}$\6
$\X3:Type definitions\X$\6
$\X2:Global variables\X$\7
$\X6:Subroutines\X$\6
$;$ $\\{main}(\\{argc},\31\\{argv})\1\1$\6
$\&{int}~\\{argc};$\6
$\&{char}~\m*\\{argv}[\,];\2\2$\6
$\a\{\1\X10:Local variables\X$\6
$;$\5
$\\{verbose}\K\\{argc}-\T{1};$\6
\&{if}~$(\\{verbose})\1$\5
$\\{sscanf}(\\{argv}[\T{1}],\31\.{"\%d"},\31\m\AND\\{spacing});\2$\6
$\X7:Initialize the data structures\X$\6
$;$\5
$\X12:Backtrack through all solutions\X$\6
$;$\5
$\\{printf}(\.{"Altogether\ \%d\ solut\)ions,\ after\ \%.15g\ upd\)ates.\\n"},%
\31\\{count},\31\\{updates});$\6
\&{if}~$(\\{verbose})\1$\5
$\X23:Print a profile of the search tree\X$\2\6
$;$\5
$\\{exit}(\T{0});\2$\6
$\}$\par
\fi

\M2. \Y\B\4$\X2:Global variables\X\EQ{}$\6
$\&{int}~\\{verbose};$\C{ $>0$ to show solutions, $>1$ to show partial ones too
}\6
$\&{int}~\\{count}\K\T{0};$\C{ number of solutions found so far }\6
$\&{double}~\\{updates};$\C{ number of times we deleted a list element }\6
$\&{int}~\\{spacing}\K\T{1};$\C{ if $\\{verbose}$, we output solutions when $%
\\{count}\MOD\\{spacing}\E\T{0}$ }\6
$\&{double}~\\{profile}[\\{max\_level}][\\{max\_degree}];$\C{ tree nodes of
given level and degree }\6
$\&{double}~\\{upd\_prof}[\\{max\_level}];$\C{ updates at a given level }\6
$\&{int}~\\{maxb}\K\T{0};$\C{ maximum branching factor actually needed }\6
$\&{int}~\\{maxl}\K\T{0};$\C{ maximum level actually reached }\par
\As 8\ET14.
\U 1.\fi

\N0 3. Data structures.
Each column of the input matrix is represented by a \&{column} struct,
and each row is represented as a linked list of \&{node} structs. There's one
node for each nonzero entry in the matrix.

More precisely, the nodes are linked circularly within each row, in
both directions. The nodes are also linked circularly within each column;
the column lists each include a header node, but the row lists do not.
Column header nodes are part of a \&{column} struct, which
contains further info about the column.

Each node contains five fields. Four are the pointers of doubly linked lists,
already mentioned; the fifth points to the column containing the node.

\Y\Y\B\4$\X3:Type definitions\X\EQ{}$\6
$\&{typedef}~\&{struct}~\\{node\_struct}$\6
$\a\{\1\&{struct}~\\{node\_struct}~\m*\\{left},\31~\m*\\{right};$\C{
predecessor and successor in row }\6
$\&{struct}~\\{node\_struct}~\m*\\{up},\31~\m*\\{down};$\C{ predecessor and
successor in column }\6
$\&{struct}~\\{col\_struct}~\m*\\{col};$\C{ the column containing this node }\2%
\6
$\}~\&{node};$\par
\A 4.
\U 1.\fi

\M4. Each \&{column} struct contains five fields:
The $\\{head}$ is a node that stands at the head of its list of nodes;
the $\\{len}$ tells the length of that list of nodes, not counting the header;
the $\\{name}$ is a one-, two-, or three-letter identifier;
$\\{next}$ and $\\{prev}$ point to adjacent columns, when this
column is part of a doubly linked list.

As backtracking proceeds, nodes
will be deleted from column lists when their row has been blocked by
other rows in the partial solution.
But when backtracking is complete, the data structures will be
restored to their original state.

\Y\B\4$\X3:Type definitions\X\PE{}$\6
\&{typedef}~\&{struct}~\\{col\_struct}\6
$\a\{\1\&{node}~\\{head};$\C{ the list header }\6
$\&{int}~\\{len};$\C{ the number of non-header items currently in this column's
list }\6
$\&{char}~\\{name}[\T{8}];$\C{ symbolic identification of the column, for
printing }\6
$\&{struct}~\\{col\_struct}~\m*\\{prev},\31~\m*\\{next};$\C{ neighbors of this
column }\2\6
$\}~\&{column};$\par
\fi

\M5. One \&{column} struct is called the root. It serves as the head of the
list of columns that need to be covered, and is identifiable by the fact
that its $\\{name}$ is empty.

\Y\B\D$\\{root}$\5
$\\{col\_array}[\T{0}$]\C{ gateway to the unsettled columns }\par
\fi

\M6. A row is identified not by name but by the names of the columns it
contains.
Here is a routine that prints a row, given a pointer to any of its
columns. It also prints the position of the row in its column.

\Y\B\4$\X6:Subroutines\X\EQ{}$\6
$\\{print\_row}( p)\1\1$\6
$\&{node}~\m* p;\2\2$\6
$\a\{$\5
\1$\&{register}~\&{node}~\m* q\K p;$\6
$\&{register}~\&{int}~ k;$\7
\&{do}\6
$\a\{\1\\{printf}(\.{"\ \%s"},\31 q\MG\\{col}\MG\\{name});$\5
$ q\K q\MG\\{right};\2$\6
$\}~\&{while}~( q\I p);$\6
\&{for}~$( q\K p\MG\\{col}\MG\\{head}.\\{down},\31 k\K\T{1};$\5
$ q\I p;$\5
$ k\PP)\1$\6
\&{if}~$( q\E\m\AND( p\MG\\{col}\MG\\{head}))$\6
$\a\{\1\\{printf}(\.{"\\n"});$\5
$\&{return};$\C{ row not in its column! }\2\6
$\}$\5
\&{else}\1\5
$ q\K q\MG\\{down};\2$\2\6
$\\{printf}(\.{"\ (\%d\ of\ \%d)\\n"},\31 k,\31 p\MG\\{col}\MG\\{len});\2$\6
$\}$\7
\&{void} $\\{print\_state}$\5
$(\1\1\1\&{int}~\\{lev}\2\2\2)$\6
$\a\{\1\&{register}~\&{int}~ l;$\7
\&{for}~$( l\K\T{0};$\5
$ l\Z\\{lev};$\5
$ l\PP)\1$\5
$\\{print\_row}(\\{choice}[ l]);\2$\2\6
$\}$\par
\As 15\ET16.
\U 1.\fi

\N0 7. Inputting the matrix.
Brute force is the rule in this part of the program.

\Y\B\4$\X7:Initialize the data structures\X\EQ{}$\6
$\X9:Read the column names\X$\6
$;$\5
$\X11:Read the rows\X$\6
$;$\par
\U 1.\fi

\M8. \Y\B\D$\\{buf\_size}$\5
$\T{4}*\\{max\_cols}+\T{3}$\C{ upper bound on input line length }\par
\Y\B\4$\X2:Global variables\X\PE{}$\6
$\&{column}~\\{col\_array}[\\{max\_cols}+\T{2}];$\C{ place for column records }%
\6
$\&{node}~\\{node\_array}[\\{max\_nodes}];$\C{ place for nodes }\6
$\&{char}~\\{buf}[\\{buf\_size}];$\par
\fi

\M9. \Y\B\D$\\{panic}( m)$\6
$\a\{$\5
\1$\\{fprintf}(\\{stderr},\31\.{"\%s!\\n\%s"},\31 m,\31\\{buf});$\5
$\\{exit}(\m-\T{1});$\5
\2$\}$\par
\Y\B\4$\X9:Read the column names\X\EQ{}$\6
$\\{cur\_col}\K\\{col\_array}+\T{1};$\5
$\\{fgets}(\\{buf},\31\\{buf\_size},\31\\{stdin});$\6
\&{if}~$(\\{buf}[\\{strlen}(\\{buf})-\T{1}]\I\.{'\\n'})\1$\5
$\\{panic}(\.{"Input\ line\ too\ long\)"});\2$\6
\&{for}~$( p\K\\{buf},\31\\{primary}\K\T{1};$\5
$\m* p;$\5
$ p\PP)$\6
$\a\{\1\&{while}~(\\{isspace}(\m* p))\1$\5
$ p\PP;\2$\6
\&{if}~$(\R\m* p)\1$\5
$\&{break};\2$\6
\&{if}~$(\m* p\E\.{'|'})$\6
$\a\{\1\\{primary}\K\T{0};$\6
\&{if}~$(\\{cur\_col}\E\\{col\_array}+\T{1})\1$\5
$\\{panic}(\.{"No\ primary\ columns"});\2$\6
$(\\{cur\_col}-\T{1})\MG\\{next}\K\m\AND\\{root},\31\\{root}.\\{prev}\K\\{cur%
\_col}-\T{1};$\5
$\&{continue};\2$\6
$\}$\6
\&{for}~$( q\K p+\T{1};$\5
$\R\\{isspace}(\m* q);$\5
$ q\PP)\1$\5
$;\2$\6
\&{if}~$( q> p+\T{7})\1$\5
$\\{panic}(\.{"Column\ name\ too\ lon\)g"});\2$\6
\&{if}~$(\\{cur\_col}\G\m\AND\\{col\_array}[\\{max\_cols}])\1$\5
$\\{panic}(\.{"Too\ many\ columns"});\2$\6
\&{for}~$( q\K\\{cur\_col}\MG\\{name};$\5
$\R\\{isspace}(\m* p);$\5
$ q\PP,\31 p\PP)\1$\5
$\m* q\K\m* p;\2$\6
$\\{cur\_col}\MG\\{head}.\\{up}\K\\{cur\_col}\MG\\{head}.\\{down}\K\m\AND\\{cur%
\_col}\MG\\{head};$\5
$\\{cur\_col}\MG\\{len}\K\T{0};$\6
\&{if}~$(\\{primary})\1$\5
$\\{cur\_col}\MG\\{prev}\K\\{cur\_col}-\T{1},\31(\\{cur\_col}-\T{1})\MG\\{next}%
\K\\{cur\_col};\2$\6
\&{else}\1\5
$\\{cur\_col}\MG\\{prev}\K\\{cur\_col}\MG\\{next}\K\\{cur\_col};\2$\6
$\\{cur\_col}\PP;\2$\6
$\}$\6
\&{if}~$(\\{primary})$\6
$\a\{\1\&{if}~(\\{cur\_col}\E\\{col\_array}+\T{1})\1$\5
$\\{panic}(\.{"No\ primary\ columns"});\2$\6
$(\\{cur\_col}-\T{1})\MG\\{next}\K\m\AND\\{root},\31\\{root}.\\{prev}\K\\{cur%
\_col}-\T{1};\2$\6
$\}$\par
\U 7.\fi

\M10. \Y\B\4$\X10:Local variables\X\EQ{}$\6
$\&{register}~\&{column}~\m*\\{cur\_col};$\6
$\&{register}~\&{char}~\m* p,\31~\m* q;$\6
$\&{register}~\&{node}~\m*\\{cur\_node};$\6
$\&{int}~\\{primary};$\par
\As 13\ET20.
\U 1.\fi

\M11. \Y\B\4$\X11:Read the rows\X\EQ{}$\6
$\\{cur\_node}\K\\{node\_array};$\6
\&{while}~$(\\{fgets}(\\{buf},\31\\{buf\_size},\31\\{stdin}))$\6
$\a\{\1\&{register}~\&{column}~\m*\\{ccol};$\6
$\&{register}~\&{node}~\m*\\{row\_start};$\7
\&{if}~$(\\{buf}[\\{strlen}(\\{buf})-\T{1}]\I\.{'\\n'})\1$\5
$\\{panic}(\.{"Input\ line\ too\ long\)"});\2$\6
$\\{row\_start}\K\NULL;$\6
\&{for}~$( p\K\\{buf};$\5
$\m* p;$\5
$ p\PP)$\6
$\a\{\1\&{while}~(\\{isspace}(\m* p))\1$\5
$ p\PP;\2$\6
\&{if}~$(\R\m* p)\1$\5
$\&{break};\2$\6
\&{for}~$( q\K p+\T{1};$\5
$\R\\{isspace}(\m* q);$\5
$ q\PP)\1$\5
$;\2$\6
\&{if}~$( q> p+\T{7})\1$\5
$\\{panic}(\.{"Column\ name\ too\ lon\)g"});\2$\6
\&{for}~$( q\K\\{cur\_col}\MG\\{name};$\5
$\R\\{isspace}(\m* p);$\5
$ q\PP,\31 p\PP)\1$\5
$\m* q\K\m* p;\2$\6
$\m* q\K\.{'\\0'};$\6
\&{for}~$(\\{ccol}\K\\{col\_array};$\5
$\\{strcmp}(\\{ccol}\MG\\{name},\31\\{cur\_col}\MG\\{name});$\5
$\\{ccol}\PP)\1$\5
$;\2$\6
\&{if}~$(\\{ccol}\E\\{cur\_col})\1$\5
$\\{panic}(\.{"Unknown\ column\ name\)"});\2$\6
\&{if}~$(\\{cur\_node}\E\m\AND\\{node\_array}[\\{max\_nodes}])\1$\5
$\\{panic}(\.{"Too\ many\ nodes"});\2$\6
\&{if}~$(\R\\{row\_start})\1$\5
$\\{row\_start}\K\\{cur\_node};\2$\6
\&{else}\1\5
$\\{cur\_node}\MG\\{left}\K\\{cur\_node}-\T{1},\31(\\{cur\_node}-\T{1})\MG%
\\{right}\K\\{cur\_node};\2$\6
$\\{cur\_node}\MG\\{col}\K\\{ccol};$\5
$\\{cur\_node}\MG\\{up}\K\\{ccol}\MG\\{head}.\\{up},\31\\{ccol}\MG\\{head}.%
\\{up}\MG\\{down}\K\\{cur\_node};$\5
$\\{ccol}\MG\\{head}.\\{up}\K\\{cur\_node},\31\\{cur\_node}\MG\\{down}\K\m\AND%
\\{ccol}\MG\\{head};$\5
$\\{ccol}\MG\\{len}\PP;$\5
$\\{cur\_node}\PP;\2$\6
$\}$\6
\&{if}~$(\R\\{row\_start})\1$\5
$\\{panic}(\.{"Empty\ row"});\2$\6
$\\{row\_start}\MG\\{left}\K\\{cur\_node}-\T{1},\31(\\{cur\_node}-\T{1})\MG%
\\{right}\K\\{row\_start};\2$\6
$\}$\par
\U 7.\fi

\N0 12. Backtracking.
Our strategy for generating all exact covers will be to repeatedly
choose always the column that appears to be hardest to cover, namely the
column with shortest list, from all columns that still need to be covered.
And we explore all possibilities via depth-first search.

The neat part of this algorithm is the way the lists are maintained.
Depth-first search means last-in-first-out maintenance of data structures;
and it turns out that we need no auxiliary tables to undelete elements from
lists when backing up. The nodes removed from doubly linked lists remember
their former neighbors, because we do no garbage collection.

The basic operation is ``covering a column.'' This means removing it
from the list of columns needing to be covered, and ``blocking'' its
rows: removing nodes from other lists whenever they belong to a row of
a node in this column's list.

\Y\B\4$\X12:Backtrack through all solutions\X\EQ{}$\6
$\\{level}\K\T{0};$\6\4
$\\{forward}$:\5
$\X19:Set $\\{best\_col}$ to the best column for branching\X$\6
$;$\5
$\\{cover}(\\{best\_col});$\5
$\\{cur\_node}\K\\{choice}[\\{level}]\K\\{best\_col}\MG\\{head}.\\{down};$\6\4
$\\{advance}$:\6
\&{if}~$(\\{cur\_node}\E\m\AND(\\{best\_col}\MG\\{head}))\1$\5
$\&{goto}~\\{backup};\2$\6
\&{if}~$(\\{verbose}>\T{1})$\6
$\a\{\1\\{printf}(\.{"L\%d:"},\31\\{level});$\5
$\\{print\_row}(\\{cur\_node});\2$\6
$\}$\6
$\X17:Cover all other columns of $\\{cur\_node}$\X$\6
$;$\6
\&{if}~$(\\{root}.\\{next}\E\m\AND\\{root})\1$\5
$\X21:Record solution and $\&{goto}~\\{recover}$\X$\2\6
$;$\5
$\\{level}\PP;$\5
$\&{goto}~\\{forward};$\6\4
$\\{backup}$:\5
$\\{uncover}(\\{best\_col});$\6
\&{if}~$(\\{level}\E\T{0})\1$\5
$\&{goto}~\\{done};\2$\6
$\\{level}\MM;$\5
$\\{cur\_node}\K\\{choice}[\\{level}];$\5
$\\{best\_col}\K\\{cur\_node}\MG\\{col};$\6\4
$\\{recover}$:\5
$\X18:Uncover all other columns of $\\{cur\_node}$\X$\6
$;$\5
$\\{cur\_node}\K\\{choice}[\\{level}]\K\\{cur\_node}\MG\\{down};$\5
$\&{goto}~\\{advance};$\6\4
$\\{done}$:\6
\&{if}~$(\\{verbose}>\T{3})\1$\5
$\X22:Print column lengths, to make sure everything has been restored\X$\2\6
$;$\par
\U 1.\fi

\M13. \Y\B\4$\X10:Local variables\X\PE{}$\6
$\&{register}~\&{column}~\m*\\{best\_col};$\C{ column chosen for branching }\6
$\&{register}~\&{node}~\m*\\{pp};$\C{ traverses a row }\par
\fi

\M14. \Y\B\4$\X2:Global variables\X\PE{}$\6
$\&{int}~\\{level};$\C{ number of choices in current partial solution }\6
$\&{node}~\m*\\{choice}[\\{max\_level}];$\C{ the row and column chosen on each
level }\par
\fi

\M15. When a row is blocked, it leaves all lists except the list of the
column that is being covered. Thus a node is never removed from a list
twice.

\Y\B\4$\X6:Subroutines\X\PE{}$\6
$\\{cover}( c)\1\1$\6
$\&{column}~\m* c;\2\2$\6
$\a\{$\5
\1$\&{register}~\&{column}~\m* l,\31~\m* r;$\6
$\&{register}~\&{node}~\m*\\{rr},\31~\m*\\{nn},\31~\m*\\{uu},\31~\m*\\{dd};$\6
$\&{register}~ k\K\T{1};$\C{ updates }\7
$ l\K c\MG\\{prev};$\5
$ r\K c\MG\\{next};$\5
$ l\MG\\{next}\K r;$\5
$ r\MG\\{prev}\K l;$\6
\&{for}~$(\\{rr}\K c\MG\\{head}.\\{down};$\5
$\\{rr}\I\m\AND( c\MG\\{head});$\5
$\\{rr}\K\\{rr}\MG\\{down})\1$\6
\&{for}~$(\\{nn}\K\\{rr}\MG\\{right};$\5
$\\{nn}\I\\{rr};$\5
$\\{nn}\K\\{nn}\MG\\{right})$\6
$\a\{\1\\{uu}\K\\{nn}\MG\\{up};$\5
$\\{dd}\K\\{nn}\MG\\{down};$\5
$\\{uu}\MG\\{down}\K\\{dd};$\5
$\\{dd}\MG\\{up}\K\\{uu};$\5
$ k\PP;$\5
$\\{nn}\MG\\{col}\MG\\{len}\MM;\2$\6
$\}$\2\6
$\\{updates}\KK+ k;$\5
$\\{upd\_prof}[\\{level}]\KK+ k;\2$\6
$\}$\par
\fi

\M16. Uncovering is done in precisely the reverse order. The pointers thereby
execute an exquisitely choreo\-graphed dance which returns them almost
magically to their former state.

\Y\B\4$\X6:Subroutines\X\PE{}$\6
$\\{uncover}( c)\1\1$\6
$\&{column}~\m* c;\2\2$\6
$\a\{$\5
\1$\&{register}~\&{column}~\m* l,\31~\m* r;$\6
$\&{register}~\&{node}~\m*\\{rr},\31~\m*\\{nn},\31~\m*\\{uu},\31~\m*\\{dd};$\7
\&{for}~$(\\{rr}\K c\MG\\{head}.\\{up};$\5
$\\{rr}\I\m\AND( c\MG\\{head});$\5
$\\{rr}\K\\{rr}\MG\\{up})\1$\6
\&{for}~$(\\{nn}\K\\{rr}\MG\\{left};$\5
$\\{nn}\I\\{rr};$\5
$\\{nn}\K\\{nn}\MG\\{left})$\6
$\a\{\1\\{uu}\K\\{nn}\MG\\{up};$\5
$\\{dd}\K\\{nn}\MG\\{down};$\5
$\\{uu}\MG\\{down}\K\\{dd}\MG\\{up}\K\\{nn};$\5
$\\{nn}\MG\\{col}\MG\\{len}\PP;\2$\6
$\}$\2\6
$ l\K c\MG\\{prev};$\5
$ r\K c\MG\\{next};$\5
$ l\MG\\{next}\K r\MG\\{prev}\K c;\2$\6
$\}$\par
\fi

\M17. \Y\B\4$\X17:Cover all other columns of $\\{cur\_node}$\X\EQ{}$\6
\&{for}~$(\\{pp}\K\\{cur\_node}\MG\\{right};$\5
$\\{pp}\I\\{cur\_node};$\5
$\\{pp}\K\\{pp}\MG\\{right})\1$\5
$\\{cover}(\\{pp}\MG\\{col});\2$\par
\U 12.\fi

\M18. We included $\\{left}$ links, thereby making the rows doubly linked, so
that columns would be uncovered in the correct LIFO order in this
part of the program. (The $\\{uncover}$ routine itself could have done its
job with $\\{right}$ links only.) (Think about it.)

\Y\B\4$\X18:Uncover all other columns of $\\{cur\_node}$\X\EQ{}$\6
\&{for}~$(\\{pp}\K\\{cur\_node}\MG\\{left};$\5
$\\{pp}\I\\{cur\_node};$\5
$\\{pp}\K\\{pp}\MG\\{left})\1$\5
$\\{uncover}(\\{pp}\MG\\{col});\2$\par
\U 12.\fi

\M19. \Y\B\4$\X19:Set $\\{best\_col}$ to the best column for branching\X\EQ{}$\6
$\\{minlen}\K\\{max\_nodes};$\6
\&{if}~$(\\{verbose}>\T{2})\1$\5
$\\{printf}(\.{"Level\ \%d:"},\31\\{level});\2$\6
\&{for}~$(\\{cur\_col}\K\\{root}.\\{next};$\5
$\\{cur\_col}\I\m\AND\\{root};$\5
$\\{cur\_col}\K\\{cur\_col}\MG\\{next})$\6
$\a\{\1\&{if}~(\\{verbose}>\T{2})\1$\5
$\\{printf}(\.{"\ \%s(\%d)"},\31\\{cur\_col}\MG\\{name},\31\\{cur\_col}\MG%
\\{len});\2$\6
\&{if}~$(\\{cur\_col}\MG\\{len}<\\{minlen})\1$\5
$\\{best\_col}\K\\{cur\_col},\31\\{minlen}\K\\{cur\_col}\MG\\{len};\2$\2\6
$\}$\6
\&{if}~$(\\{verbose})$\6
$\a\{\1\&{if}~(\\{level}>\\{maxl})$\6
$\a\{\1\&{if}~(\\{level}\G\\{max\_level})\1$\5
$\\{panic}(\.{"Too\ many\ levels"});\2$\6
$\\{maxl}\K\\{level};\2$\6
$\}$\6
\&{if}~$(\\{minlen}>\\{maxb})$\6
$\a\{\1\&{if}~(\\{minlen}\G\\{max\_degree})\1$\5
$\\{panic}(\.{"Too\ many\ branches"});\2$\6
$\\{maxb}\K\\{minlen};\2$\6
$\}$\6
$\\{profile}[\\{level}][\\{minlen}]\PP;$\6
\&{if}~$(\\{verbose}>\T{2})\1$\5
$\\{printf}(\.{"\ branching\ on\ \%s(\%d\))\\n"},\31\\{best\_col}\MG\\{name},%
\31\\{minlen});\2$\2\6
$\}$\par
\U 12.\fi

\M20. \Y\B\4$\X10:Local variables\X\PE{}$\6
$\&{register}~\&{int}~\\{minlen};$\6
$\&{register}~\&{int}~ j,\31~ k,\31~ x;$\par
\fi

\M21. \Y\B\4$\X21:Record solution and $\&{goto}~\\{recover}$\X\EQ{}$\6
$\a\{\1\\{count}\PP;$\6
\&{if}~$(\\{verbose})$\6
$\a\{\1\\{profile}[\\{level}+\T{1}][\T{0}]\PP;$\6
\&{if}~$(\\{count}\MOD\\{spacing}\E\T{0})$\6
$\a\{\1\\{printf}(\.{"\%d:\\n"},\31\\{count});$\6
\&{for}~$( k\K\T{0};$\5
$ k\Z\\{level};$\5
$ k\PP)\1$\5
$\\{print\_row}(\\{choice}[ k]);\2$\2\6
$\}$\2\6
$\}$\6
$\&{goto}~\\{recover};\2$\6
$\}$\par
\U 12.\fi

\M22. \Y\B\4$\X22:Print column lengths, to make sure everything has been
restored\X\EQ{}$\6
$\a\{\1\\{printf}(\.{"Final\ column\ length\)s"});$\6
\&{for}~$(\\{cur\_col}\K\\{root}.\\{next};$\5
$\\{cur\_col}\I\m\AND\\{root};$\5
$\\{cur\_col}\K\\{cur\_col}\MG\\{next})\1$\5
$\\{printf}(\.{"\ \%s(\%d)"},\31\\{cur\_col}\MG\\{name},\31\\{cur\_col}\MG%
\\{len});\2$\6
$\\{printf}(\.{"\\n"});\2$\6
$\}$\par
\U 12.\fi

\M23. \Y\B\4$\X23:Print a profile of the search tree\X\EQ{}$\6
$\a\{\1\&{double}~\\{tot},\31~\\{subtot};$\7
$\\{tot}\K\T{1};$\C{ the root node doesn't show up in the profile }\6
\&{for}~$(\\{level}\K\T{1};$\5
$\\{level}\Z\\{maxl}+\T{1};$\5
$\\{level}\PP)$\6
$\a\{\1\\{subtot}\K\T{0};$\6
\&{for}~$( k\K\T{0};$\5
$ k\Z\\{maxb};$\5
$ k\PP)$\6
$\a\{\1\\{printf}(\.{"\ \%5.6g"},\31\\{profile}[\\{level}][ k]);$\5
$\\{subtot}\KK+\\{profile}[\\{level}][ k];\2$\6
$\}$\6
$\\{printf}(\.{"\ \%5.15g\ nodes,\ \%.15\)g\ updates\\n"},\31\\{subtot},\31%
\\{upd\_prof}[\\{level}-\T{1}]);$\5
$\\{tot}\KK+\\{subtot};\2$\6
$\}$\6
$\\{printf}(\.{"Total\ \%.15g\ nodes.\\n\)"},\31\\{tot});\2$\6
$\}$\par
\U 1.\fi

\N0 24. Index.
\fi


\inx
\@m\\{advance}, \[12].
\@m\\{argc}, \[1].
\@m\\{argv}, \[1].
\@m\\{backup}, \[12].
\@m\\{best\_col}, 12, \[13], 19.
\@m\\{buf}, \[8], 9, 11.
\@m\\{buf\_size}, \[8], 9, 11.
\@m c, \[15], \[16].
\@m\\{ccol}, \[11].
\@m\\{choice}, 6, 12, \[14], 21.
\@m\\{col}, \[3], 6, 11, 12, 15, 16, 17, 18.
\@m\\{col\_array}, 5, \[8], 9, 11.
\@h\&{col\_struct}, \[4].
\@h\&{column}, \[4], 5, 8, 10, 11, 13, 15, 16.
\@m\\{count}, 1, \[2], 21.
\@m\\{cover}, 12, \[15], 17.
\@m\\{cur\_col}, 9, \[10], 11, 19, 22.
\@m\\{cur\_node}, \[10], 11, 12, 17, 18.
\@m\\{dd}, \[15], \[16].
\@m\\{done}, \[12].
\@m\\{down}, \[3], 6, 9, 11, 12, 15, 16.
\@m\\{exit}, 1, 9.
\@m\\{fgets}, 9, 11.
\@m\\{forward}, \[12].
\@m\\{fprintf}, 9.
\@m\\{head}, \[4], 6, 9, 11, 12, 15, 16.
\@m\\{isspace}, 9, 11.
\@m j, \[20].
\@m k, \[6], \[15], \[20].
\@m l, \[6], \[15], \[16].
\@m\\{left}, \[3], 11, 16, 18.
\@m\\{len}, \[4], 6, 9, 11, 15, 16, 19, 22.
\@m\\{lev}, \[6].
\@m\\{level}, 12, \[14], 15, 19, 21, 23.
\@m\\{main}, \[1].
\@m\\{max\_cols}, \[1], 8, 9.
\@m\\{max\_degree}, \[1], 2, 19.
\@m\\{max\_level}, \[1], 2, 14, 19.
\@m\\{max\_nodes}, \[1], 8, 11, 19.
\@m\\{maxb}, \[2], 19, 23.
\@m\\{maxl}, \[2], 19, 23.
\@m\\{minlen}, 19, \[20].
\@m\\{name}, \[4], 5, 6, 9, 11, 19, 22.
\@m\\{next}, \[4], 9, 12, 15, 16, 19, 22.
\@m\\{nn}, \[15], \[16].
\@h\&{node}, \[3], 4, 6, 8, 10, 11, 13, 14, 15, 16.
\@m\\{node\_array}, \[8], 11.
\@m\\{node\_struct}, \[3].
\@m p, \[6], \[10].
\@m\\{panic}, \[9], 11, 19.
\@m\\{pp}, \[13], 17, 18.
\@m\\{prev}, \[4], 9, 15, 16.
\@m\\{primary}, 9, \[10].
\@m\\{print\_row}, \[6], 12, 21.
\@m\\{print\_state}, \[6].
\@m\\{printf}, 1, 6, 12, 19, 21, 22, 23.
\@m\\{profile}, \[2], 19, 21, 23.
\@m q, \[6], \[10].
\@m r, \[15], \[16].
\@m\\{recover}, \[12], 21.
\@m\\{right}, \[3], 6, 11, 15, 17, 18.
\@m\\{root}, \[5], 9, 12, 19, 22.
\@m\\{row\_start}, \[11].
\@m\\{rr}, \[15], \[16].
\@m\\{spacing}, 1, \[2], 21.
\@m\\{sscanf}, 1.
\@m\\{stderr}, 9.
\@m\\{stdin}, 9, 11.
\@m\\{strcmp}, 11.
\@m\\{strlen}, 9, 11.
\@m\\{subtot}, \[23].
\@m\\{tot}, \[23].
\@m\\{uncover}, 12, \[16], 18.
\@m\\{up}, \[3], 9, 11, 15, 16.
\@m\\{upd\_prof}, \[2], 15, 23.
\@m\\{updates}, 1, \[2], 15.
\@m\\{uu}, \[15], \[16].
\@m\\{verbose}, 1, \[2], 12, 19, 21.
\@m x, \[20].
\fin
\@$\X12:Backtrack through all solutions\X$
\U 1.
\@$\X17:Cover all other columns of $\\{cur\_node}$\X$
\U 12.
\@$\X2, 8, 14:Global variables\X$
\U 1.
\@$\X7:Initialize the data structures\X$
\U 1.
\@$\X10, 13, 20:Local variables\X$
\U 1.
\@$\X23:Print a profile of the search tree\X$
\U 1.
\@$\X22:Print column lengths, to make sure everything has been restored\X$
\U 12.
\@$\X9:Read the column names\X$
\U 7.
\@$\X11:Read the rows\X$
\U 7.
\@$\X21:Record solution and $\&{goto}~\\{recover}$\X$
\U 12.
\@$\X19:Set $\\{best\_col}$ to the best column for branching\X$
\U 12.
\@$\X6, 15, 16:Subroutines\X$
\U 1.
\@$\X3, 4:Type definitions\X$
\U 1.
\@$\X18:Uncover all other columns of $\\{cur\_node}$\X$
\U 12.
\con
